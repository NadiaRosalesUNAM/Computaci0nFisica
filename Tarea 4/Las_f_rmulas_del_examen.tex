\documentclass[letterpaper,12pt]{article}
\usepackage[utf8]{inputenc}
\usepackage{geometry}
\usepackage[dvipsnames]{xcolor}
\usepackage{mathrsfs}
\usepackage{amsmath}
\usepackage{amssymb}
\usepackage{latexsym}

\usepackage{fancyhdr}
\pagestyle{fancy}
\fancyhf{}
\lhead{\leftmark}
\rhead{\textit{Nadia Estefania Rosales Orozco}}
\cfoot{\Large{pág.\thepage}}

\author{Nadia Rosales Orozco }
\date{27 de octubre de 2022}

\begin{document}

\textbf{\Large{\textcolor{Green}{¿CÓMO QUE NO PUEDO SACAR MI ACORDEÓN?}}}

\section{\textit{Cinemática}}
\begin{itemize}
    \item [\#]
    \small{Velocidad: es el cambio de posición en un intervalo de tiempo.}
    \begin{equation*}%entorno matemático
    \vec{V}=\frac{\Delta\vec{r}}{\Delta\vec{t}}
\end{equation*}
\end{itemize}

\section{\textit{Dinámica de fluidos}}%no le quito la numeración porque sino no aparece en el encabezado
    \begin{itemize}
    \item [\&]
    \small{Ecuación de Bernoulli: relaciona la presión, la velocidad y la altura de dos puntos cualesquiera (1 y 2) en un fluido con flujo laminar constante de densidad.}
    \begin{equation*}%entorno matemático
    \textcolor{red}{P_1+\frac{1}{2}v^2_1\rho+\rho gh_1=P_2+\frac{1}{2}v^2_2\rho+\rho gh_2}
\end{equation*}
\end{itemize}

\section{\textit{Equilibrio químico}}

\section{\textit{Termodinámica}}

\section{\textit{Fenómenos de transporte}}

\end{document}
