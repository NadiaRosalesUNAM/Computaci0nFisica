\documentclass[letterpaper,12pt]{article}
\usepackage[utf8]{inputenc}
\usepackage{geometry}
\usepackage[dvipsnames]{xcolor}
\usepackage{mathrsfs}
\usepackage{amsmath}
\usepackage{amssymb}
\usepackage{latexsym}
\usepackage{pifont}

\usepackage{fancyhdr}
\pagestyle{fancy}
\fancyhf{}
\lhead{\leftmark}
\rhead{\textit{Nadia Estefania Rosales Orozco}}
\cfoot{\Large{pág.\thepage}}

\author{Nadia Rosales Orozco }
\date{27 de octubre de 2022}

\begin{document}

\textbf{\Large{\textcolor{Green}{¿CÓMO QUE NO PUEDO SACAR MI ACORDEÓN?}}}

\section{\textit{Cinemática y dinámica}}

\begin{itemize}
    \item [\ding{46}]
    \small{Velocidad media: es el cambio de posición en un intervalo de tiempo.}
    \end{itemize}
    \begin{equation*}%entorno matemático
    \vec{V}=\frac{\Delta\vec{r}}{\Delta{t}}=\frac{\vec{r_f}-\vec{r_i}}{t_f-t_i}
    \end{equation*}
    
    \begin{itemize}
    \item [\ding{40}]
    \small{Aceleración media: es el cambio de la velocidad en un intervalo de tiempo.}
    \end{itemize}
    \begin{equation*}%entorno matemático
    \vec{a}=\frac{\Delta\vec{V}}{\Delta{t}}=\frac{\vec{V_f}-\vec{V_i}}{t_f-t_i}
    \end{equation*}
    
    \begin{itemize}
    \item [\ding{119}]
    \small{Posición con movimiento rectilíneo uniformemente acelerado (MRUA): donde $x_0$ es la posición inicial, $v_o$ es la velocidad inicial, t es el tiempo y a es la aceleración.}
    \end{itemize}
    \begin{equation*}%entorno matemático
    \textcolor{blue}{x(t)=x_0+v_0t+\frac{1}{2}at^2}
    \end{equation*}
    
    \begin{itemize}
    \item [\ding{46}]
    \small{Segunda Ley de Newton: establece que la rapidez con la que cambia el momento lineal es igual a la resultante de las fuerzas que actúan sobre él, donde $\Sigma\vec{F}$ representa la fuerza total que actúa sobre el cuerpo, $\Delta\vec{p}$ es la variación del momento lineal y $\Delta t$ es el intervalo de tiempo.}
    \end{itemize}
    \begin{equation*}%entorno matemático
    \vec{F}=\frac{\Delta\vec{p}}{\Delta{t}}
    \end{equation*}
    
     \begin{itemize}
    \item [\ding{40}]
    \small{Momento angular: momento de la cantidad de movimiento de una partícula, es decir, el producto vectorial de su vector de posición por su momento lineal, donde $\vec{L}$ es el momento angular o cinético del cuerpo, $\vec{r}$ es el vector de posición del cuerpo respecto al punto O, y $\vec{p}$ es la cantidad de movimiento del cuerpo.}
    \end{itemize}
    \begin{equation*}%entorno matemático
    \vec{L}=\vec{r}X\vec{p}=\vec{r}xm\cdot \vec{v}
    \end{equation*}\vspace{3cm}
    
\section{\textit{Dinámica de fluidos}}%no le quito la numeración porque sino no aparece en el encabezado
    \begin{itemize}
    \item [\ding{106}]
    \small{Ecuación de Bernoulli: relaciona la presión, la velocidad y la altura de dos puntos cualesquiera (1 y 2) en un fluido con flujo laminar constante de densidad $\rho$}.
    \end{itemize}
    \begin{equation*}%entorno matemático
    \textcolor{red}{P_1+\frac{1}{2}v^2_1\rho+\rho gh_1=P_2+\frac{1}{2}v^2_2\rho+\rho gh_2}
\end{equation*}

     \begin{itemize}
    \item [\ding{109}]
    \small{Ecuaciones de Euler: describen el movimiento de un fluido compresible no viscoso. El sistema queda formado por las ecuaciones de la conservación de masa (1), cantidad de movimiento (2) y conservación de la energía (3). La energía total por unidad de masa es $e$ y $h$ la entalpía}.
    \end{itemize}
    \begin{align}
    \frac{\delta p}{\delta t}+\nabla\cdot(\rho V)=&0\\
    \frac{\delta (\rho V)}{\delta t}+\nabla\cdot(\rho V\otimes  V)=&0\\
    \frac{\delta (\rho e)}{\delta t}+\nabla\cdot(\rho hV)=&0
    \end{align}

    \begin{itemize}
    \item [\ding{126}]
    \small{Presión hidrostática: presión que se somete un cuerpo sumergido en un fluido, debido a la columna de líquido que tiene sobre él.}.
    \end{itemize}
    \begin{equation*}
    P_h=\rho gh
\end{equation*}

\section{\textit{Equilibrio químico}}
\begin{itemize}
    \item [\ding{171}]
    \small{Constante de equilibrio: se expresa como la relación entre las concentraciones molares (mol/l) de reactivos y productos. Su valor en una reacción química depende de la temperatura la cual debe especificarse. $aA+bB\rightleftarrows fF+eE\hspace{1cm}$}
    \end{itemize}
    \begin{equation*}
    \textcolor{Red}{K=\frac{[F]^f[E]^e}{[A]^a[B]^b}}
    \end{equation*}
    
    \begin{itemize}
    \item [\ding{118}]
    \small{Energía libre de Gibbs: es el cambio de energía de Gibbs es la variación de energía libre de un sistema, de un estado termodinámico a otro. El valor de la variación de la energía libre representa la máxima cantidad de energía liberada, la cual, puede ser utilizada por el sistema (absorbida) al ir de un estado inicial a un estado final.}
    \end{itemize}
    \begin{equation*}
    \Delta G=\Delta H-T\Delta S
    \end{equation*}
    
\begin{itemize}
    \item [\ding{101}]
    \small{Ecuación de Nernst: se utiliza para calcular el potencial de reducción de un electrodo fuera de las condiciones estándar.}
    \end{itemize}
    \begin{equation*}%entorno matemático
    E=E^O-\frac{RT}{nF}lnQ
    \end{equation*}

\section{\textit{Estadística y probabilidad}}

    \begin{itemize}
    \item [\ding{41}]
    \small{Desviación media: es la media aritmética de los valores absolutos de las desviaciones respecto a la media.}
    \end{itemize}
    $$D\bar{x}=\frac{\sum\limits_{i=1}^{N}|x_i-x|}{N}$$

    \begin{itemize}
    \item [\ding{171}]
    \small{Desviación estándar poblacional: es la raíz cuadrada de la varianza, es una medida de la dispersión de los datos.}
    \end{itemize}
    \begin{equation*}%entorno matemático
    \textcolor{Orange}{S=\sqrt{\frac{\sum\limits_{i=1}^{N}{(\bar{X}-x_i)}^2}{N}}}
    \end{equation*}

    \begin{itemize}
    \item [\ding{40}]
    \small{Distribución binomial: es una distribución de probabilidad discreta que cuenta el número de éxitos en una secuencia de n ensayos de Bernoulli independientes entre sí con una probabilidad fija p de ocurrencia de éxito entre los ensayos. Se emplea en experimentos con repetición, la probabilidad de éxito es siempre la misma, donde n es el número de veces que se repite el experimento, x es el número de éxitos esperados y p es la probabilidad de éxito.}
    \end{itemize}
    \begin{equation*}%entorno matemático
    P(x,n,p)=
    \begin{pmatrix}
    n\\
    x\\
    \end{pmatrix}*p^x*(1-p)^{n-x}
    \end{equation*}
    
    \begin{itemize}
    \item [\ding{45}]
    \small{Distribución Poisson: es una distribución de probabilidad discreta que expresa, a partir de una frecuencia de ocurrencia media, la probabilidad de que ocurra un determinado número de eventos durante cierto período de tiempo. La población es grande y la probabilidad de éxito es pequeña. Donde $\lambda$ es el promedio dado por $\lambda=N*p$, x es el número de éxitos esperados, p es la probabilidad de éxito y N es el tamaño de la población.}
    \end{itemize}
    \begin{equation*}%entorno matemático
    P(x,y)=\frac{e^{-\lambda}*\lambda^x}{x!}
    \end{equation*}
    
    \begin{itemize}
    \item [\ding{99}]
    \small{Cuartiles para datos agrupados: son medidas de posición relativa. El primer cuartil Q1, es el valor en el cual o por debajo del cual queda aproximadamente un cuarto (25\%) de todos los valores de la sucesión (ordenada); El segundo cuartil Q2 es el valor por debajo del cual queda el 50\% de los datos (Mediana), el tercer cuartil Q3 es el valor por debajo del cual quedan las tres cuartas partes (75\%) de los datos. Donde L es el límite inferior de clase, N es la suma de frecuencias absolutas, $F_{i-1}$ es la frecuencia acumulada anterior a la clase y $a_i$ es la amplitud de clase.}
    \end{itemize}
    \begin{equation*}%entorno matemático
    Q_k=L_i+\frac{\frac{kN}{4}-F_{i-1}}{f_i}a_i
    \end{equation*}

\section{\textit{Fenómenos de transporte}}
    \begin{itemize}
    \item [\ding{79}]
    \small{Ley de Fick: la velocidad de difusión a través de una membrana es directamente proporcional al gradiente de concentración de la sustancia a ambos lados de la misma e inversamente proporcional al grosor de la membrana.}
    \end{itemize}
    \begin{equation*}
    \textcolor{red}{J_{AY}=-D_{AB}\frac{dC_A}{dy}}
    \end{equation*}
    
    \begin{itemize}
    \item [\ding{81}]
    \small{Ley de Fourier: existe una proporcionalidad entre el flujo de energía J y el gradiente de temperatura.}
    \end{itemize}
    \begin{equation*}
    q_{y}=-K\frac{dT}{dy}
    \end{equation*}

    \begin{itemize}
    \item [\ding{82}]
    \small{Ley de la viscosidad de Newton: cuando a un fluido se le aplica una fuerza o un esfuerzo cortante, el fluido presenta una resistencia al movimiento, conforme continúa dicho esfuerzo el fluido tiende a deformarse. Posteriormente fluye y su velocidad aumenta conforme aumenta el esfuerzo.}
    \end{itemize}
    \begin{equation*}
    \tau_{yx}=-\mu \frac{dV_x}{dy}
    \end{equation*}

\end{document}
