\documentclass[a5paper,11pt]{article}
\usepackage[utf8]{inputenc}
\usepackage[left=3cm,right=2.5cm,top=1.6cm,bottom=1.6cm]{geometry}
\usepackage[dvipsnames]{xcolor}

\color{white}

\usepackage{graphicx}
\usepackage{caption}
\usepackage[right]{sidecap}

\usepackage{fancyhdr}
\pagestyle{fancy}
\fancyhf{}
\chead{\tiny{\leftmark}}
\lfoot{\small{Nadia Estefania Rosales Orozco}}
\rfoot{\small{\thepage}}

\author{nadiarosaleso }
\date{22 de semptiembre de 2022}

\begin{document}
\pagecolor{CadetBlue}

\begin{center}
\textbf{\Huge{\textcolor{Goldenrod}{\underline{Mi serie favorita}}}}
\end{center}

\section{\sc{Avatar la leyenda de Aang\\¿Película?, no hay no existe}}
    \subsection*{\sf{\textit{\textcolor{SpringGreen}{Hace muchos años las cuatro naciones vivían en armonía...}}}}
    \textsf{\textcolor{black}{La serie se sitúa en un mundo dividido en cuatro grandes naciones, la Nación del Fuego, las Tribus Agua, los Nómadas Aire y el Reino Tierra. Muchas personas nacen con la habilidad de dominar uno de los elementos, pero sólo una de ellas puede dominar los cuatro elementos {\it{agua, tierra, fuego y aire}}, el {\it{Avatar}}, éste se encarga de mantener la paz entre las cuatro naciones y también del mundo físico con el mundo espiritual.\\ \\ Después de la muerte del avatar Roku,originario de la Nación del Fuego, el señor del fuego Sozin decidió atacar los templos de los nómadas aire para evitar la próxima reencarnación del avatar y que su nación se convirtiera en la dominante. Antes el inicio de la guerra el avatar Aang, nacido en el Templo del Aire del Sur, escapó de su hogar con su bisonte volador y mejor amigo Appa, terminó en una tormenta, pero gracias al espíritu del avatar entró en un estado de sueño dentro de un iceberg. Cien años después es encontrado por Katara (maestra agua) y Sokka, dos aldeanos de la Tribu del agua del Sur. Sin saberlo llaman la atención del príncipe desterrado de la Nación del Fuego, el príncipe Zuko, que junto con su tío Iroh viaja por el mundo para encontrar al avatar y recuperar su honor. \\ \\Aang, Katara y Sokka logran escapar del príncipe y viajan por todo el mundo para encontrar un maestro en la Tribu agua del Norte, en el camino conocen a nuevos amigos, entre ellos, Momo, Bumi y Suki. Despúes de dominar el agua Aang debe dominar la Tierra y encuentra a su maestra, la niña ciega, pero la mejor maestra de su tipo Toph Beifong. Perseguidos por Zuko, y por otro lado por su hermana Azula y sus amigas, los cuatro protagonistas buscan un maestro fuego para Aang, el cual debe derrotar al señor del fuego para recuperar la paz en el mundo. Dos antagonistas se unen por caminos del destino, el príncipe Zuko y el avatar Aang y su pandilla. Logran restaurar la paz y entrar a una nueva era que traerá grandes cambios políticos y sociales.\\ \\Esta es una fantástica serie llena de personajes inolvidables, momentos cómicos, tristes, y épicos; mensajes importantes como la amistad, el perdón, la persistencia y el honor. Aunque es una serie para todo público, no deja de lado temas importantes como son el imperialismo, el genocidio, la muerte, la destrucción del ambiente, y la pérdida de identidad cultural. Además, es un mundo tan extenso que se realizó una segunda serie, Avatar la leyenda de Korra, centrada 60 años después con nuevos conflictos en un mundo entrando en plena revolución tecnológica. La franquicia también cuenta con una gran cantidad de cómics, miniseries, y fans por todo el globo.}}
    
    \subsection*{\sf{\textit{\textcolor{SpringGreen}{De la voz}}}}
    \textcolor{black}{\begin{enumerate}
    \item Creadores
            \begin{enumerate}
                \item Michael Dante DiMartino
                \item Bryan Konietzko
            \end{enumerate}
        \item Estudio de animación
            \begin{enumerate}
                \item Nickelodeon
            \end{enumerate}
        \item Estudio de doblaje y dirección
            \begin{enumerate}
                \item Estudio DINT Doblajes Internacionales
                \item Alexis Quiroz
                \item Yaninna Quiroz
                \item Traducción Natalia Valdebenito
            \end{enumerate}
        \item Actores de doblaje en español principales
            \begin{enumerate}
                \item Aang: René Pinochet
                \item Katara: Jessica Toledo
                \item Sokka: Sergio Aliaga
                \item Zuko: Pablo Ausensi
                \item Toph Beifong: Ximena Marchant
                \item Iroh: Mario Santander
            \end{enumerate}
        \item Actores de doblaje en español secundarios
        \begin{enumerate}
            \item Azula: Yaninna Quiroz
            \item Suki: Maureen Herman
            \item Rey Bumi: Javier Rodríguez
            \item Ozai: Daniel Seisdedos
            \item Mai: Viviana Navarro
            \item Ty Lee: Vanesa Silva
        \end{enumerate}
    \end{enumerate}}
    \subsection*{\sf{\textit{\textcolor{SpringGreen}{Los actores de la isla Ember}}}}
    \textcolor{black}{\begin{itemize}
        \item [\|] \fcolorbox{Maroon}{SkyBlue}{\textcolor{Blue}{Aang:}} Avatar nacido en el Templo del aire del Sur, sucesor del avatar Roku.
        \item [\|] \fcolorbox{Maroon}{SkyBlue}{\textcolor{Blue}{Katara:}} Última maestra agua de la Tribu agua del sur.
        \item [\|] \fcolorbox{Maroon}{SkyBlue}{\textcolor{Blue}{Sokka:}} Guerrero, nacido en la tribu del agua del sur, hermano de Katara.
        \item [\|] \fcolorbox{Maroon}{SkyBlue}{\textcolor{Blue}{Toph:}} La mejor maestra Tierra, primera en aprender metal control.
         \item [\|] \fcolorbox{Maroon}{SkyBlue}{\textcolor{Blue}{Zuko:}} Príncipe desterrado y heredero de la Nación del fuego.
          \item [\|] \fcolorbox{Maroon}{SkyBlue}{\textcolor{Blue}{Iroh:}} General, conocido como el dragón del Oeste, tío de Zuko.
    \end{itemize}}
    
    \subsection*{\sf{\textit{\textcolor{SpringGreen}{Se busca}}}}

    \begin{figure}[h]
        \begin{flushright}
        \caption*{\hspace{2cm}{Avatar la leyenda de Aang}}
        \includegraphics[scale=0.28,angle=15]{Imágenes/Avatar la leyenda de Aang.png}
        \end{flushright}
    \end{figure}

\section{\sc{Como pocos mundos ficticios}}
    \subsection*{\sf{\textit{\textcolor{Apricot}{¿Why?}}}}
    \textsf{\textcolor{black}{{\textcolor{Yellow}{Avatar la leyenda de Aang}} es una de mis series favoritas. No había visto otra serie con personajes tan complejos y únicos, un universo extenso, trama original y bien desarrollada, artes marciales, arte de animación, música y mensajes no vacíos; hasta que llegó {\textcolor{CarnationPink}{Arcane}} trece años después. Tiene algunos de los personajes más entrañables de la animación, principalmente el {\textcolor{Orange}{príncipe Zuko}}, que pasó una vida turbulenta pero simpre mantuvo su idealismo.}}
    \subsection*{\sf{\textit{\textcolor{Apricot}{El mejor y el peor}}}}
     \begin{SCfigure}[]
        \caption{Príncipe Zuko, heredero del legado del avatar Roku y del señor del fuego Sozin}
        \includegraphics[scale=0.5,angle=-18]{Imágenes/Evolución del príncipe Zuko.PNG}
    \end{SCfigure}
    
     \begin{SCfigure}[]
        \caption{Viejo traidor *\#*\$* de la temporada uno que traicionó a Katara y Haru}
        \includegraphics[scale=0.8,angle=8]{Imágenes/Este desgraciado.PNG}
    \end{SCfigure}

\vspace{-3cm}{\sf{\textit{\tiny{\hspace{-2.8cm}``El orgullo no es lo\vspace}}}}
\sf{\textit{\tiny{\vspace{}{\hspace{-3.3cm}opuesto de la vergüenza\vspace}}}}
\sf{\textit{\tiny{\vspace{}{\hspace{-3.3cm}sino su fuente;\vspace}}}}
\sf{\textit{\tiny{\vspace{}{\hspace{-3.3cm}la humildad pura es el\vspace}}}}
\sf{\textit{\tiny{\vspace{}{\hspace{-3.3cm}único antídoto para la\vspace}}}}
\sf{\textit{\tiny{\vspace{}{\hspace{-3.3cm}vergüenza"Iroh\vspace}}}
}

\end{document}
